\documentclass{beamer}

\usetheme{Madrid}
\usecolortheme{beaver}

\title{Igra na srečo: Igralni avtomat}
\subtitle{Projektna naloga pri predmetu Matematika z računalnikom}
\author{Justin Raišp}
\date{20. januar 2026}

\begin{document}

\begin{frame}
    \titlepage
\end{frame}

\begin{frame}{Koncept in ideja}
    Cilj projekta je bila izdelava funkcionalnega digitalnega igralnega avtomata, ki združuje:
    \begin{itemize}
        \item \textbf{Matematično ozadje:} Izračun verjetnosti in donosa,
        \item \textbf{Simulacijo:} Preverjanje teoretičnih izračunov z velikim številom vrtljajev,
        \item \textbf{Uporabniški vmesnik:} Dinamičen prikaz igre v Pythonu.
    \end{itemize}
\end{frame}
\begin{frame}{Pravila igre in mehanika izplačil}
    \textbf{Osnovna pravila:}
    \begin{itemize}
        \item Igra poteka na mreži $3 \times 5$ z 20 plačilnimi linijami.
        \item Dobitne kombinacije se štejejo izključno od leve proti desni, začenši s prvim kolutom.
        \item Izplačila se izračunajo kot $\frac{\text{Stava}x\text{Množitelj simbola}}{\text{Število plačilnih linij}}$
    \end{itemize}
    \begin{figure}
        \centering
        \includegraphics[width=0.42\textwidth]{Figures/izplacilo0.png}
        \caption{Primer dobitne linije s tremi simboli.}
    \end{figure}
\end{frame}
\begin{frame}{Posebni simboli in izračun dobitkov}
    \textbf{Posebni simboli:}
    \begin{itemize}
        \item \textbf{Wild:} Nadomešča vse simbole, razen scatter in cashpot, za sestavo najboljše kombinacije.
        \item \textbf{Cashpot:} Posebni simboli z naključnimi vrednostmi od 0,2 do 100, katerih vsota se izplača le v primeru, ko jih je 5 ali več. Izplačajo večkratnik celotne stave,
        \item \textbf{Scatter:} 3 ali več simboli kjerkoli na zaslonu sprožijo bonus igro s \textbf{5 brezplačnimi vrtljaji}, kjer nastopajo le cashpot simboli,
    \end{itemize}
\end{frame}
\begin{frame}
    \begin{figure}
        \centering
          \begin{minipage}{0.45\textwidth}
            \centering
            \includegraphics[width=\textwidth]{Figures/izplacilo.png}
            \caption{Dobitna linija z Wild simbolom.}
        \end{minipage}
        \hfill
        \begin{minipage}{0.45\textwidth}
            \centering
            \includegraphics[width=\textwidth]{Figures/casshpot.png}
            \caption{Dobitek s Cashpot simboli.}
        \end{minipage}
    \end{figure}
\end{frame}
\begin{frame}{Matematični izračuni}
    \textbf{Return to Player (RTP)} je statistična mera, ki izraža odstotek celotnega vplačanega zneska, ki ga igralni avtomat dolgoročno izplača igralcem.
    
    \vfill
    \textbf{Matematična formula:}
    $$ RTP = \frac{\sum \text{Dobitki}}{\sum \text{Vplačila}} \times 100 \% $$
    Pred programiranjem je bilo treba določiti matematični model v Excelu:
    \begin{itemize}
        \item \textbf{Vrednosti simbolov:} Določiti porazdelitve izplačil posameznih simbolov,
        \item Pripraviti dva različna \textbf{reelset-a}, enega z višjim in drugega z nižjim RTP ter ju ustrezno utežiti,
        \item Pripraviti porazdelitev večkratnikov za cashpot simbole,
        \item Ustrezno pripraviti izplačila v bonus igri.
    \end{itemize}
\end{frame}

\begin{frame}
  \begin{figure}
      \centering
      \includegraphics[width=1\textwidth]{Figures/izracun.png}
      \caption{Primer izračuna v Excelu.}
  \end{figure}
\end{frame}

\begin{frame}{Simulacije}
    \begin{itemize}
        \item Izvedba \textbf{400 000 000} avtomatiziranih vrtljajev.
        \item \textbf{Analiza rezultatov:}
            \begin{itemize}
                \item Primerjava dejanskega RTP s teoretičnim,
                \item Preverjanje delež izplačila po posameznih simbolih,
                \item Preverjanje pogostosti bonus igre in izplačila le te.
            \end{itemize}
    \end{itemize}
    S pomočjo Studentove porazdelitve sem dobil 95\% interval zapupanja in s tem potrdil izračune.
\end{frame}


\begin{frame}{Uporabniški vmesnik}
    Vmesnik je zasnovan v knjižnici \textbf{Tkinter} in vključuje:
    \begin{itemize}
        \item \textbf{Glavno mrežo:} dinamično osveževanje slik simbolov,
        \item \textbf{Interaktivne elemente:} gumbi za stavo, vrtenje in info stran,
        \item \textbf{Info stran:} tabela dobitkov in prikaz 20 plačilnih linij.
    \end{itemize}
\end{frame}

\begin{frame}{Zaključek}
    Projekt uspešno združuje matematično teorijo verjetnosti s praktično uporabo v programiranju. 
    
    \vfil
    \textbf{Glavna spoznanja:}
    \begin{itemize}
        \item Način računanja verjetnosti in izplačil v igralnih avtomatih s pomočjo excela,
        \item Uporaba simulacij za potrjevanje teoretičnih modelov,
        \item Ustvarjanje intuitivnega grafičnega vmesnika za boljšo uporabniško izkušnjo.
    \end{itemize}
\end{frame}

\end{document}